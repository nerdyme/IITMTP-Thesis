\chapter{Epilogue}

\section{Challenges}
\begin{enumerate}
\item Multi-Lingual Support for the local language spoken by community people.
\item Design in coherence with end user capabilities to minimize user input on the GUI and increase its usability.
\item Capturing user requirements
\item Content Management and Content Moderation
\item Understanding the design structure of Gram Vaani, the methology of managing surveys, their mapping and creation.
\item Data security and data moderation regarding the information of the users
\item Scalability of the application
\end{enumerate}

\section{Conclusions}

\begin{enumerate}
\item Main issues and problems are specific to the villages, to the regions. They
need to be identified and then application can be used and make a great
contribution in actually helping people by various means.
\item Manual intervention and involvements are the key elements in introducing
big changes and turning heads of the people.
\item The local knowledge of village is very important prior introducing any new
model in that place.
\item Necessity of responsible people in various regulatory authorities, commission
departments, panchayats, Government officers, NGO’s workers, ASHA
workers, school teachers.
\item Application is designed for both the Hindi and English application users.
\item Web Portal interface is designed for the NGO personnels.
\item The surveys and other functionalities will help in dissemination of information effectively.
\end{enumerate}


\section{Future Work}

\begin{enumerate}
\item \textbf{Deployment of Mobile Application} - Mobile Application will be deployed in the community where volunteers will be provided with the pre loaded .apk file with the android phones. Volunteers’ feedback will be analysed and incorporated to increase the usability of the application. Volunteers should be given timely trainings regarding the usage of application and other uses  of basic phone must be taught.
\item \textbf{Creating Surveys from Application} - Feature of creating a new survey, adding questions, uploading audio prompts etc  can be added in the application for volunteers as well as on the portal for the NGO users.
\item \textbf{Sending Announcements and Messages from Portal} - NGO user can record audio announcements and can send instant text messages to the volunteers on their mobile phones. 
\end{enumerate}




